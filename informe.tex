\documentclass{article}
\usepackage{graphicx}
\begin{document}
\begin{figure}
\centering
\includegraphics[width=0.9\linewidth]{./logo_fiuba_alta}
\label{fig:logo_fiuba_alta}
\end{figure}

\title{Informe sobre GO - Teor\'{\i}a del Lenguaje}
\author{Cristian Gonzalez \and Billy}
\date{1° Cuatrimestre, 2015}
\maketitle
\section{Historia}
\section{Recolector de Basura}
El Recolector de Basura de Go usa una variante de \textit{marcado-y-barrido} donde en vez de usar solamente 2 colores para el marcado, usa 3. 
Ac\'{a} explicamos la características de cada conjunto.
\\
\\


Otra caracteristica importante es \textit{stop-the-world} el cual pausa, al momento de correr el recolector, todos los hilos menos aquellos que utilice el recolector hasta que finalice el mismo. 

\end{document}